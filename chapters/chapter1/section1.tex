\section{روش‌های \lr{Anti-Disassembly}}
\label{sec:ch1-sec1}

\subsection{مقدمه ای بر ضد دی‌اسمبل (\lr{Anti-Disassembly})}

در چرخه تکامل نرم‌افزار، حفاظت از مالکیت فکری و امنیت کد منبع به عنوان یکی از چالش‌های اساسی توسعه‌دهندگان و سازمان‌ها مطرح است. با گسترش روزافزون تهدیدات امنیتی و افزایش تمایل به مهندسی معکوس (\lr{Reverse Engineering}) برای اهداف مختلف - از تحلیل رقابتی گرفته تا استخراج آسیب‌پذیری‌ها و ایجاد بدافزار - نیاز به مکانیزم‌های محافظتی مؤثر بیش از پیش احساس می‌شود.

روش‌های ضد دی‌اسمبل (\lr{Anti-Disassembly - AD}) به عنوان یکی از کارآمدترین تکنیک‌ها در حوزه اُبفوسکیشن (\lr{Obfuscation}) شناخته می‌شوند. هدف اصلی این روش‌ها، ایجاد موانع ساختاریافته در مسیر تحلیلگران و ابزارهای خودکار مهندسی معکوس است. \lr{AD} با دستکاری عمدی ساختار کد اجرایی (\lr{Executable})، سعی در گمراه کردن دی‌اسمبلرها و تخریب صحت نمودار جریان کنترل (\lr{Control Flow Graph - CFG}) دارد.

اهمیت به‌کارگیری تکنیک‌های \lr{AD} تنها محدود به محافظت در برابر سرقت کد نیست؛ بلکه در زمینه‌های امنیتی مانند محافظت از الگوریتم‌های احراز هویت، جلوگیری از دستکاری نرم‌افزار (\lr{Tampering})، و افزایش هزینه‌های تحلیل بدافزارها نیز نقش حیاتی ایفا می‌کند. در محیط‌های رقابتی امروزی، استفاده از \lr{AD} می‌تواند زمان و منابع مورد نیاز برای درک منطق کسب‌وکار (\lr{Business Logic}) یک نرم‌افزار را به میزان قابل توجهی افزایش دهد.

در این فصل، به بررسی جامع روش‌های ضد دی‌اسمبل، مبانی نظری، اهداف، و دسته‌بندی‌های مختلف آن خواهیم پرداخت. ابتدا تعریف دقیقی از \lr{AD} و تفاوت آن با سایر روش‌های اُبفوسکیشن ارائه می‌شود. سپس، اهمیت و کاربردهای \lr{AD} در چارچوب درس تکامل نرم‌افزار بررسی می‌گردد. در ادامه، به تشریح تکنیک‌های کلیدی \lr{AD} از جمله استفاده از دستورات نامعتبر، پرش‌های شرطی پیچیده، و دستکاری اشاره‌گرهای دستورات پرداخته و تأثیر هر یک بر تحلیل استاتیک مورد بحث قرار می‌گیرد. درک این مفاهیم، پایه‌ای اساسی برای فصل آتی که به مقابله با این تکنیک‌ها (\lr{Anti-Anti-Disassembly}) می‌پردازد، فراهم خواهد ساخت.