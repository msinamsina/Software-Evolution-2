\section{استفاده از تکنیک‌های تغییر ساختار کد}
\label{sec:ch1-obfuscation}

مهندسی معکوس، به مجموعه‌ای از تکنیک‌ها اطلاق می‌شود که به تحلیل‌گر اجازه می‌دهد با استفاده از ابزارهایی مانند دیس‌اسمبلرها و دی‌کامپایلرها، فایل اجرایی را به یک نمایش سطح پایین‌تر یا حتی یک بازسازی نزدیک به کد منبع اصلی تبدیل کند. این امر درهای سوءاستفاده را به روی افراد مخرب باز می‌کند؛ از جمله دور زدن مکانیزم‌های حفاظتی نرم‌افزار، سرقت مالکیت معنوی برای ساخت محصولات رقیب، و شناسایی آسیب‌پذیری‌های امنیتی برای بهره‌برداری‌های بعدی.

به منظور ایجاد یک مانع جدی در برابر این تهدیدات، از رویکردی به نام مبهم‌سازی کد استفاده می‌شود. مبهم‌سازی فرآیندی است که طی آن، کد برنامه به نسخه‌ای دیگر تبدیل می‌شود که از نظر عملکردی کاملاً با نسخه اصلی یکسان است، اما درک، تحلیل و دنبال کردن منطق آن برای یک انسان یا حتی ابزارهای خودکار، به شدت دشوار و پیچیده می‌گردد. هدف نهایی این تکنیک، غیرممکن ساختن مطلق مهندسی معکوس نیست، بلکه افزایش هزینه، زمان و سطح تخصص مورد نیاز برای آن است، تا جایی که این فرآیند برای مهاجم از نظر عملی و اقتصادی فاقد صرفه باشد. تکنیک‌های مبهم‌سازی را می‌توان در دسته‌بندی‌های مختلفی مورد بررسی قرار داد.

\subsection{دسته‌بندی تکنیک‌های مبهم‌سازی کد}
این تکنیک‌ها بر اساس جنبه‌ای از برنامه که هدف قرار می‌دهند، به سه دسته اصلی تقسیم می‌شوند: مبهم‌سازی واژگانی، مبهم‌سازی داده‌ها و مبهم‌سازی جریان کنترل.

\subsubsection{۱. مبهم‌سازی واژگانی}
این دسته از تکنیک‌ها بر روی ساختار سطحی و نوشتاری کد و اطلاعات مرتبط با آن تمرکز دارند و به عنوان اولین لایه دفاعی عمل می‌کنند.

\paragraph{تغییر نام شناسه‌ها:}
این روش یکی از اساسی‌ترین و در عین حال مؤثرترین تکنیک‌ها است. توسعه‌دهندگان به صورت طبیعی از اسامی معنادار برای اجزای مختلف کد مانند متغیرها، توابع و کلاس‌ها استفاده می‌کنند تا خوانایی و قابلیت نگهداری کد را تضمین کنند. این اسامی، سرنخ‌های معنایی بسیار ارزشمندی را در اختیار تحلیل‌گر قرار می‌دهند. تکنیک تغییر نام، به صورت خودکار این اسامی بامفهوم را با شناسه‌های کوتاه، بی‌معنی و غیرقابل پیش‌بینی جایگزین می‌کند. در نتیجه، کدی که توسط دی‌کامپایلر بازسازی می‌شود، فاقد هرگونه زمینه معنایی است و تحلیل‌گر را مجبور می‌سازد تا برای درک عملکرد هر بخش، با صرف زمان و انرژی بسیار زیاد، جریان اجرای برنامه را به صورت دستی ردیابی کند و نقش هر عنصر را حدس بزند.

\paragraph{حذف اطلاعات فراداده‌ای و اشکال‌زدایی:}
فایل‌های اجرایی، به خصوص در پلتفرم‌های مدیریت‌شده، اغلب حاوی اطلاعات اضافی به نام فراداده هستند. این اطلاعات شامل ساختار کلی برنامه، نام کلاس‌ها، فضاهای نام و همچنین اطلاعات اشکال‌زدایی است که می‌تواند مسیر تحلیل را برای مهندس معکوس بسیار هموار کند. حذف این اطلاعات غیرضروری، تحلیل‌گر را از یک دید کلی و ساختاریافته نسبت به برنامه محروم کرده و او را مجبور به تحلیل در سطح پایین‌تر و جزئی‌تر می‌کند.

\subsubsection{۲. مبهم‌سازی داده‌ها}
این تکنیک‌ها بر نحوه ذخیره‌سازی، نمایش و ساختار داده‌ها در برنامه تمرکز دارند تا فهم مقادیر و روابط بین آن‌ها را مختل کنند.

\paragraph{رمزنگاری رشته‌ها:}
رشته‌های متنی ثابت که در کد برنامه وجود دارند، اهداف اولیه‌ی تحلیل‌گران هستند. این تکنیک، تمام این رشته‌ها را در زمان کامپایل به فرمتی غیرقابل خواندن تبدیل کرده و در فایل اجرایی ذخیره می‌کند. در زمان اجرا، تنها در لحظه‌ای که برنامه به یک رشته نیاز دارد، یک روتین مخصوص آن را به حالت اولیه بازمی‌گرداند. این فرآیند باعث می‌شود که تحلیل‌گر نتواند با یک جستجوی ساده به اطلاعات حساس دست یابد و ابتدا باید مکانیزم رمزگشایی را شناسایی و مهندسی معکوس کند.

\paragraph{تبدیل ساختار داده‌ها:}
ساختارهای داده منطقی مانند کلاس‌ها، داده‌های مرتبط را به صورت یکپارچه سازماندهی می‌کنند. این تکنیک این پیوندهای منطقی را از هم می‌گسلد. به عنوان مثال، یک ساختار داده‌ای منسجم می‌تواند به چندین آرایه جداگانه و نامرتبط تبدیل شود. این امر درک مدل داده‌ای برنامه و نحوه ارتباط اجزای مختلف داده با یکدیگر را برای تحلیل‌گر فوق‌العاده دشوار می‌سازد.

\paragraph{کدگذاری متغیرها:}
به جای ذخیره‌سازی مقدار یک متغیر به صورت مستقیم، مقدار آن تحت یک تبدیل منطقی یا حسابی قرار گرفته و به آن شکل ذخیره می‌شود. هر زمان که برنامه نیاز به خواندن یا نوشتن مقدار واقعی متغیر داشته باشد، ابتدا باید تبدیل معکوس آن را اجرا کند. این کار باعث می‌شود که مشاهده مستقیم مقادیر متغیرها در حافظه در حین فرآیند دیباگ، اطلاعات مفیدی را در اختیار تحلیل‌گر قرار ندهد، زیرا مقادیر مشاهده شده، مقادیر واقعی نیستند.

\subsubsection{۳. مبهم‌سازی جریان کنترل}
این دسته از تکنیک‌ها، که از پیچیده‌ترین و قوی‌ترین روش‌ها هستند، منطق و مسیر اجرای دستورات برنامه را هدف قرار می‌دهند.

\paragraph{درج کدهای بی‌اثر و گزاره‌های مبهم:}
در این روش، دستورات و بلوک‌های کدی به برنامه اضافه می‌شوند که هیچ تأثیر واقعی بر خروجی نهایی آن ندارند. شکل پیشرفته‌تر این تکنیک، استفاده از گزاره‌های مبهم است. یک گزاره مبهم، یک عبارت شرطی است که نتیجه آن همواره ثابت است، اما به گونه‌ای طراحی شده که ابزارهای تحلیل استاتیک قادر به تشخیص این ماهیت ثابت نباشند. این گزاره‌ها برای ایجاد انشعاب‌های کنترلی جعلی در برنامه استفاده می‌شوند که تحلیل‌گر انسانی و ابزارهای خودکار را به مسیرهای تحلیلی اشتباه و بی‌نتیجه هدایت می‌کنند.

\paragraph{صاف‌سازی جریان کنترل:}
این یک تکنیک بسیار قدرتمند است که ساختارهای منطقی و قابل فهم برنامه مانند حلقه‌ها و دستورات شرطی را به طور کامل از بین می‌برد. در این روش، بدنه یک تابع به بلوک‌های کد کوچکتر تقسیم می‌شود. سپس تمام این بلوک‌ها در داخل یک ساختار کنترلی مرکزی قرار می‌گیرند. یک متغیر حالت تعیین می‌کند که در هر لحظه کدام بلوک کد باید اجرا شود و هر بلوک پس از اجرا، این متغیر حالت را برای تعیین بلوک بعدی به‌روزرسانی می‌کند. نتیجه، یک ساختار اجرایی شبیه به کلاف سردرگم است که فاقد هرگونه جریان منطقی قابل تشخیص بوده و دنبال کردن آن برای تحلیل‌گر تقریباً غیرممکن است.
