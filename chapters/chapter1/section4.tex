\section{پنهان‌سازی از طریق تجزیه و تحلیل پویا}

پنهان‌سازی از طریق تجزیه و تحلیل پویا یکی از پیشرفته‌ترین و کارآمدترین شیوه‌های مقابله با مهندسی معکوس است؛ زیرا بخش‌هایی از منطق برنامه تنها در زمان اجرا و تحت شرایط مشخص قابل مشاهده یا تحلیل هستند. این ویژگی باعث می‌شود ابزارهای تحلیل استاتیک نتوانند ساختار واقعی برنامه را بازسازی کنند و تحلیل‌گر مجبور به استفاده از روش‌های پویا گردد \cite{kruegel2006rootkits}.

یکی از مهم‌ترین تکنیک‌ها در این حوزه، رمزگشایی زمان اجرا \en{(Runtime Decryption)} است. در این روش بخش‌های حساس برنامه به‌صورت رمزگذاری‌شده ذخیره شده و تنها هنگام نیاز رمزگشایی می‌شوند. پس از اجرای تابع نیز این بخش‌ها مجدداً پاک شده یا دوباره رمزگذاری می‌شوند، که این موضوع دسترسی تحلیل‌گر به نسخهٔ کامل و پایدار کد را دشوار می‌کند \cite{moser2007limits}. این روش در بدافزارهای پیچیده و نرم‌افزارهای تجاری با حساسیت بالا بسیار رایج است.

روش دیگری که در پنهان‌سازی پویا کاربرد گسترده دارد، بارگذاری پویا \en{(Dynamic Code Loading)} است. در این تکنیک، بخش‌هایی از برنامه در فایل اجرایی وجود ندارند و در زمان اجرا از حافظهٔ رمزگذاری‌شده، منابع بیرونی یا حتی شبکه بارگذاری می‌شوند. این ویژگی باعث بی‌اثر شدن بخش قابل توجهی از تحلیل‌های استاتیک می‌شود و تحلیل‌گر را مجبور به تحلیل پویا و ثبت رفتار اجرای زنده می‌کند \cite{egele2008survey}.

از سوی دیگر، تکنیک‌های تغییر مسیر کنترل در زمان اجرا \en{(Dynamic Control Flow Modification)} نیز با ایجاد وابستگی مسیر اجرای برنامه به شرایط محیطی مانند مقدار رجیسترها، وجود دیباگر، زمان‌بندی اجرای دستورها یا وضعیت سخت‌افزار، فرآیند تحلیل را برای مهاجم پیچیده می‌کنند. این وابستگی باعث می‌شود مسیر واقعی اجرا تنها در بستر واقعی قابل مشاهده باشد و ابزارهای تحلیلی قادر به بازسازی آن نباشند \cite{kruegel2006rootkits}.

بخش مهمی از پنهان‌سازی پویا به‌کارگیری تکنیک‌های ضد دیباگ پویا \en{(Dynamic Anti-Debugging)} است. روش‌هایی مانند اندازه‌گیری تأخیر اجرای دستورها، بررسی \en{Breakpoint}ها، شناسایی \en{Emulator}، یا فراخوانی توابع خاص سیستم‌عامل جهت تشخیص شرایط غیرعادی اجرا—همگی برای جلوگیری از تحلیل کد توسط ابزارهای دیباگ استفاده می‌شوند. در صورت تشخیص چنین شرایطی، برنامه می‌تواند منطق اصلی خود را پنهان کرده یا مسیر اجرا را تغییر دهد و تحلیل‌گر را گمراه کند \cite{kwon2014antidebug}.

با وجود پتانسیل بالا، پنهان‌سازی پویا محدودیت‌هایی نیز دارد. ابزارهای تحلیل پیشرفته با استفاده از تکنیک‌هایی مانند حافظه‌برداری \en{(Memory Dumping)} و ابزارهای درج پویا \en{(Dynamic Instrumentation)} می‌توانند نسخه‌هایی از کد رمزگشایی‌شده در حافظه را استخراج کنند \cite{doland2016instrumentation}. با این حال، بهره‌گیری هم‌زمان از چندین روش پویا و ایستا همچنان یکی از مقاوم‌ترین راهبردهای مقابله با مهندسی معکوس محسوب می‌شود.