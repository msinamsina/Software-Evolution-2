\section{استفاده از دیس اسمبلرهای پیشرفته}
\subsection{دلیل نام‌گذاری پیشرفته برای این نوع اسمبلر ها}
دی‌اسمبلرهای مدرن (مانند \lr{IDA Pro} و \lr{Ghidra}) فراتر از یک ابزار ساده برای ترجمه کد ماشین به اسمبلی عمل می‌کنند و به همین دلیل عنوان "پیشرفته" را به خود اختصاص داده‌اند. این برتری به دلیل سه قابلیت کلیدی زیر است:
\begin{enumerate}
\item \textbf{تحلیل هوشمند و سطح بالا (\lr{High-Level Analysis})}
\begin{itemize}
\item \textbf{دکامپایلر داخلی}
مهم‌ترین تفاوت، وجود دکامپایلر است. این قابلیت به ابزار اجازه می‌دهد کد اسمبلی سطح پایین را به یک شبه کد سطح بالا (شبیه زبان \lr{C/C++}) تبدیل کند، که درک منطق برنامه را ده‌ها برابر سریع‌تر و آسان‌تر می‌کند.
\item \textbf{ترسیم \lr{CFG}}
این ابزارها قادر به ترسیم دقیق نمودار جریان کنترل (\lr{Control Flow Graph}) هستند.
\end{itemize}
\item \textbf{قابلیت مقابله با موانع امنیتی (\lr{AADA Integration})}
\begin{itemize}
\item \textbf{اسکریپت‌نویسی قوی}
دی‌اسمبلرهای پیشرفته دارای \lr{API} برنامه‌نویسی (مانند \lr{IDAPython}) هستند که به تحلیلگر اجازه می‌دهد کارهای تکراری را خودکار کند.
\item \textbf{خنثی‌سازی \lr{Anti-Disassembly (AADA)}}
اسکریپت‌ها برای شناسایی الگوهای مبهم‌سازی (\lr{Obfuscation}) و اعمال تصحیحات خودکار (مانند \lr{Patching}) در کد استفاده می‌شوند تا موانع امنیتی را خنثی کنند.
\end{itemize}
\item \textbf{یکپارچگی و انعطاف‌پذیری}
\begin{itemize}
\item \textbf{پشتیبانی چند معماری}
آن‌ها از ده‌ها معماری پردازنده (\lr{ARM}، \lr{MIPS}، \lr{x86/x64}) پشتیبانی می‌کنند، برخلاف ابزارهای قدیمی که محدود بودند.
\item \textbf{مدیریت پروژه}
تمامی نتایج تحلیل، نام‌گذاری‌ها، و تصحیحات انجام شده در یک پایگاه داده پروژه ذخیره می‌شود و نیازی به تحلیل مجدد در هر بار بازگشایی نیست.
\end{itemize}
\end{enumerate}
به طور خلاصه، دی‌اسمبلرهای پیشرفته به دلیل تبدیل شدن به محیط‌های هوشمند، خودکار و یکپارچه برای تحلیل عمیق کد، و توانایی آن‌ها در غلبه بر دفاعیات \lr{Anti-Disassembly}، ابزارهای قدیمی را پشت سر گذاشته‌اند.