\chapter*{چکیده}
\addcontentsline{toc}{chapter}{چکیده}

گزارش حاضر با هدف تحلیل معماری دفاعی نرم‌افزار، به بررسی روش‌های متقابل در برابر تحلیل کدهای اجرایی می‌پردازد. با توجه به اهمیت حیاتی جلوگیری از مهندسی معکوس و سوءاستفاده از مالکیت فکری و آسیب‌پذیری‌های امنیتی، توسعه‌دهندگان به سمت استفاده از تکنیک‌های ضد دیس‌اسمبلی (AD) سوق داده شده‌اند.

در بخش نخست، تکنیک‌های اصلی AD مورد بررسی قرار می‌گیرد که شامل:

ابهام‌سازی کد (Code Obfuscation) در سه سطح واژگانی، داده‌ای، و جریان کنترل (با استفاده از روش‌هایی مانند صاف‌سازی جریان کنترل و درج کدهای بی‌اثر).

بهره‌گیری از دستورات و ساختارهای غیرمعمول و وابسته به حالت پردازنده، از جمله دستورات FPU و کدهای خود-تغییردهنده، برای شکستن تحلیل استاتیک دیس‌اسمبلرها.

روش‌های پنهان‌سازی مبتنی بر تجزیه و تحلیل پویا، مانند رمزگشایی زمان اجرا (Runtime Decryption) و استفاده از تکنیک‌های ضد دیباگ، که کد اصلی را تنها در زمان اجرای زنده آشکار می‌سازند.

تکنیک‌های رمزگذاری برای مقاومت در برابر دیس‌اسمبلی، از جمله کدهای پلی‌مورفیک (Polymorphic Code) و متامورفیک (Metamorphic Code).

در بخش دوم، گزارش به فلسفه و تکنیک‌های ضد ضد دیس‌اسمبلی (AADA) می‌پردازد که در پاسخ به پیچیدگی‌های AD توسط تحلیل‌گران امنیتی ایجاد شده‌اند. تکنیک‌های کلیدی AADA شامل:

تحلیل دینامیک و رصد کد (Dynamic Analysis and Tracing) برای کشف مسیر اجرای واقعی و استخراج آدرس‌های مقصد پرش‌های محاسباتی.

استفاده از دیس‌اسمبلرهای پیشرفته (مانند IDA Pro و Ghidra) با قابلیت‌هایی چون دکامپایلر داخلی، تحلیل هوشمند جریان داده، و اسکریپت‌نویسی قوی برای خودکارسازی فرآیند خنثی‌سازی.

بهره‌گیری از اجرای نمادین (Symbolic Execution) و شبیه‌سازی داینامیک برای کاوش در مسیرهای اجرایی مبهم و تحلیل کدهای خود-تغییر.

وصله‌زدن و اصلاح کد (Patching) برای تبدیل پرش‌های داینامیک به استاتیک و بازسازی نمودار جریان کنترل (CFG).

در نهایت، نتایج نشان می‌دهند که مقابله با مهندسی معکوس یک «مسابقه تسلیحاتی» دائمی میان روش‌های دفاعی AD و روش‌های تحلیلی AADA است. موفقیت در حفظ امنیت و تکامل نرم‌افزار، مستلزم درک عمیق از هر دو جبهه و استفاده هوشمندانه و ترکیبی از ابزارهای تحلیل استاتیک و دینامیک برای خنثی‌سازی مؤثر لایه‌های ابهام‌سازی است.
\newpage

