\chapter*{چکیده}
\addcontentsline{toc}{chapter}{چکیده}

تحولات مهندسی نرم‌افزار از روش‌های ابتدایی بدون ساختار به مدل‌های خطی مانند آبشاری و سپس به رویکردهای تکرارشونده و چابک، نشان‌دهنده تلاش برای مدیریت پیچیدگی و افزایش کیفیت سیستم‌ها بوده است. با این حال، چالش‌هایی مانند ارتباطات ناکارآمد، مستندسازی ضعیف، انباشت بدهی فنی و ناسازگاری با فناوری‌های نوین، همچنان تهدیدی برای پایداری نرم‌افزارها محسوب می‌شوند.

DevOps به‌عنوان فرهنگی نوین و مجموعه‌ای از ابزارها، با هدف یکپارچه‌سازی تیم‌های توسعه و عملیات، افزایش سرعت تحویل و ارتقای کیفیت معرفی شد. خودکارسازی فرآیندهای CI/CD و استفاده از ابزارهایی مانند Docker، Kubernetes و Jenkins، امکان تحویل سریع، کاهش خطا و افزایش پایداری استقرار را فراهم می‌کند. با این حال، چالش‌هایی مانند پیچیدگی زیرساخت و مقاومت فرهنگی نیازمند آموزش، مستندسازی و فرهنگ‌سازی هستند.

بازطراحی نرم‌افزار برای سیستم‌های قدیمی ضروری است و دلایل آن شامل ضعف معماری، فناوری‌های منسوخ، تغییر نیازمندی‌ها و انباشت بدهی فنی است. تکنیک‌های بازآرایی ، مهندسی معکوس و مهاجرت افزایشی \en{(Incremental Migration)} ضمن افزایش قابلیت نگهداری، هزینه اصلاح بدهی فنی را کاهش می‌دهند و ریسک تغییرات را مدیریت می‌کنند. مطالعات موردی مانند بازطراحی اپلیکیشن PayPal و نئوبانک فوربیکس، نشان‌دهنده تاثیر مستقیم بازطراحی بر تجربه کاربری، یکپارچگی خدمات و مزیت رقابتی هستند.

در نهایت، ترکیب رویکردهای DevOps و بازطراحی، پایداری و تکامل نرم‌افزارها را تضمین می‌کند. توصیه می‌شود تیم‌های مهندسی بر فرهنگ همکاری، مدیریت فعال بدهی فنی، مستندسازی همگام با توسعه، خودکارسازی CI/CD و آموزش مستمر تمرکز کنند. همچنین، مسیرهای تحقیقاتی آینده شامل امنیت یکپارچه در DevOps (DevSecOps)، کاربرد هوش مصنوعی در مهندسی معکوس، مدیریت پیچیدگی ابزارها، آموزش مهارت‌های نرم و توسعه الگوهای پیشرفته مهاجرت و بازطراحی خواهد بود.


\newpage

